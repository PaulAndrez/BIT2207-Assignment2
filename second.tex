\documentclass{article}
\raggedbottom
\begin{document}

\title{A SIMPLE OBJECT ORIENTED LANGUAGE}
\date{16-April-2017}
\author{SOME NIGGhs}

\maketitle
\newpage

\section{Background}
Object oriented programming is used to provide a more dynamic method of running code at runtime using objects to select the procedural code to run. Today, many object oriented languages exist with different capabilities which can be used to determine what language to use when developing a program. However, no object oriented language has been specified to directly deal with mathematical formulae at the simplest level and with maximum optimisation. The language that we are to design: MathF, shall be used to make programs that can solve mathematical problems with the highest level of ease and optimisation.

\section{Main Objective}
Development of a simple object oriented language to simplify complex mathematical equations and formulae and a compiler for the language.

\section{Specific Objectives}
\begin{itemize}
\item
The domain of the language and its mathematical principles
\item
The object oriented principles
\item
The lexical rules of the language
\item
The syntactical rules of the language
\item
The semantic rules of the language
\item
Data types, control statements and operators
\item
Exception handling
\item
Runtime libraries
\item
The compiler design
\end{itemize}

\section{Methodology}
The project will be carried out by a team of four students. The team will go through multiple discussions on the possible words, syntax and semantics of the language. The object oriented principles will also have to be discussed and implemented within the rules of the language. UML (Unified Modeling Language) principles and diagrams will be used in the analysis of the language rules. Binary trees will also be used to break down the language into smaller and manageable syntax for further analysis. A compiler will be designed to translate the new language into machine language.

\end{document}